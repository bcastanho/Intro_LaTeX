\documentclass{beamer}

\usepackage[utf8]{inputenc}
\usepackage[english,russian]{babel}
\usepackage{amsmath}
\usepackage{tabularx}
\usepackage{verbatim}
\usepackage{listings}
\usepackage{tikz}
\usetikzlibrary{shapes,arrows,positioning,shadows}
\usenavigationsymbolstemplate{}

\newcommand{\cmd}[1]{{\color[HTML]{008000}\bftt{#1}}}


\title{Introduction to \LaTeX}
\author{Bruno Castanho Silva}

\date{27.07.2017}

\subtitle{Day 1: Basics}

\begin{document}

\begin{frame}
\titlepage
\end{frame}

 
\begin{frame}{What Is \LaTeX?}
\begin{itemize}
	\item A document preparation system.
	\item Plain text instead of WYSIWYG.
	\item \cmd{commands} are used to define structure and style, as well as add citations, figures, tables, or equations.
	\item A \TeX distribution (TeX Live or MikTeX) turns the plain text into PDF outputs.
\end{itemize} 
\end{frame}
 
\begin{frame}{How does it work?}
		\begin{itemize}
			\item You write your document in \texttt{plain text} with \cmd{commands} that
			describe its structure and meaning.
			\item The \texttt{latex} program processes your text and commands to produce a
			beautifully formatted document.
			\end{itemize}
\end{frame} %% end{frame} must start at the beginning of the line, otherwise error...
	
\begin{frame}[fragile]{Or...}
		\begin{verbatim}
			\begin{frame}{How does it work?}
				\begin{itemize}
					\item You write your document in \texttt{plain text} with 
					\cmd{commands} that describe its structure and meaning.
					\item The \texttt{latex} program processes your text and 
					commands to produce a beautifully formatted document.
				\end{itemize}
			\end{frame}
			\end{verbatim}
\end{frame}

\begin{frame}{What do We Need?}
	\begin{itemize}
		\item A text-editor
		\begin{itemize}
			\item Notepad will do. Or Vi, Emacs,...
		\end{itemize}
		\item A TeX distribution (compiler)
		\begin{itemize}
			\item MikTeX or TeX Live
		\end{itemize} 
		\item A PDF Viewer
		\begin{itemize}
			\item Adobe Reader, Google Chrome, Document Viewer, etc
		\end{itemize}
		\item Some GUI's have all three built in, facilitating the workflow:
		\begin{itemize}
			\item E.g.: TeXStudio. Online: Sharelatex
		\end{itemize}
	\end{itemize}
\end{frame}

\begin{frame}[fragile]{Hello, world!}
	\begin{itemize}
		\item Open a new file in TeXStudio.
		\item Write the following block of code:
			\end{itemize}
		\begin{verbatim}
		\documentclass{article}
		\begin{document}
		Hello, world!
		\end{document}
		\end{verbatim}
		\begin{itemize}
			\item Hit the double green arrow on top.
		\end{itemize}
\end{frame}
	
\begin{frame}[fragile]{Unpacking...}
	\begin{itemize}
		\item Commands always start with a backslash: \textbackslash
		\item The first command (\verb=\documentclass{article}=) defines the kind of document we are producing.
		\begin{itemize}
			\item There are a few different kinds to choose from. Each one has default formatting options.
			\item Examples include \cmd{article}, \cmd{report}, \cmd{proc}, \cmd{letter}, or \cmd{plain}. More on that later...
		\end{itemize}
		\item Between \{ \} you include what you want that command applied to.
		\begin{itemize}
			\item E.g.: to make a word \textbf{bold}, you put it between \{\} in the command \verb=\textbf{}=.
		\end{itemize}
		\item The PDF output will only show text written in the \cmd{document} environment, meaning, between \verb=\begin{document}= and \verb=\end{document}=.
		\item Everything before that is called the \textit{Preamble}.
 	\end{itemize}
\end{frame}	

\begin{frame}[fragile]
	\begin{itemize} 
	\item Quick exercise: Open the file called \cmd{Rock paper.tex} in Sharelatex. Starting with your blank document, produce the following sentence (with formatting) in an output (tip, the command for italic is \verb=\textit{}=):
	\begin{itemize}
		\item Scissors cuts paper, paper covers rock, rock crushes \textbf{lizard}, \textbf{lizard} poisons \textit{Spock}, \textit{Spock} smashes scissors, scissors decapitates \textbf{lizard}, \textbf{lizard} eats paper, paper disproves \textit{Spock}, \textit{Spock} vaporizes rock, and as it always has, rock crushes scissors.
	\end{itemize}
	\end{itemize}
\end{frame}

\begin{frame}[fragile]{Special Characters}
	\begin{itemize}
		\item Some special characters have coding functions in \LaTeX, so if you type them, they will not show up...
		\begin{itemize}
			\item We already saw about \textbackslash, that starts all commands.
			\item Others are: \{ \} \$ \& \# \% \textasciitilde \textasciicircum \_
		\end{itemize}
		\item Code to produce the line above:
	\end{itemize}
	\begin{verbatim}
	\item Others are: \{ \} \$ \& \# 
	\% \textasciitilde \textasciicircum \_
	\end{verbatim}
\end{frame}

\begin{frame}[fragile]{Spaces}
	\begin{itemize}
		\item \TeX~treats multiple spaces in your plain text as a single space
		\item Therefore, ``\verb=I    saw a    tiger='' is compiled as ``\verb=I saw a tiger=''.
		\begin{itemize}
			\item But but I want to have triple spaces!
			\begin{itemize}
				\item Use \textasciitilde~to force a space. Example:
				\item ``\verb=I~~~~~want spaces='' compiles as ``I~~~~~want spaces''.
			\end{itemize}
		\end{itemize} 
	\end{itemize}
		
	\begin{itemize}
		\item Breaks between paragraphs are marked by a blank line in the text file. If you use a single Enter, it will be compiled as a single line.
	\end{itemize}
\end{frame}

\begin{frame}{To produce...}
	Happy families are all alike; every unhappy family is unhappy in its own way.
	
	Everything was in confusion in the Oblonskys' house. The wife had discovered that the husband was carrying on an intrigue with a French girl, who had been a governess in their family, and she had announced to her husband that she could not go on living in the same house with him.
\end{frame}

\begin{frame}[fragile]{Tolstoi typed...}
	\begin{lstlisting}[breaklines]
	Happy families are all alike; every unhappy family is unhappy in its own way.
	
	Everything was in confusion in the Oblonskys' house. The wife had discovered that the husband was carrying on an intrigue with a French girl, who had been a governess in their family, and she had announced to her husband that she could not go on living in the same house with him.
	\end{lstlisting}
\end{frame}

\begin{frame}[fragile]{Instead of...}
	\begin{lstlisting}[breaklines]
	Happy families are all alike; every unhappy family is unhappy in its own way.
	Everything was in confusion in the Oblonskys' house. The wife had discovered that the husband was carrying on an intrigue with a French girl, who had been a governess in their family, and she had announced to her husband that she could not go on living in the same house with him.
	\end{lstlisting}
\end{frame}

\begin{frame}{Exercise}
\begin{itemize}
	\item In the \cmd{Bonnie.tex} file in ShareLaTeX, add the following formatting to the paragraph.
\end{itemize}
	\textbf{Bonnie \& Clyde} stole \$45,000 dollars from the \#1 bank in the country, Fathers \& Sons.
	
	That was \textit{85\% of the bank's reserves}, and they went bankrupt shortly after.
\end{frame}

\begin{frame}[fragile]{Solution}
	\begin{lstlisting}[breaklines]
	\textbf{Bonnie \& Clyde} stole \$45,000 dollars from the \#1 bank in the country, Fathers \& Sons.
	
	That was \textit{85\% of the bank's reserves}, and they went bankrupt shortly after.
	\end{lstlisting}
\end{frame}

\begin{frame}[fragile]{More Useful Characters}
	\begin{itemize}
		\item Quotation marks won't work. You need one (or two) backtick (\`{}) on the left, and apostrophes (') on the right. So...
		\item \`{}\`{}Really? No quotation marks?\'{}\'{}~ produces ``Really? No quotation marks?''.
		\item \% is used for comments to your code. If you simply type \%, everything after it in a line will not be printed. 
		\item Example. Typing:
	\end{itemize}
	\begin{verbatim}
	Blablabla % should we cite the crappy Gabor paper here??
	\end{verbatim}
	prints
	
	
	\bigskip
	
	Blablabla % should we cite the crappy Gabor paper here??
\end{frame}

\begin{frame}{Errors}
	\begin{itemize}
		\item With so many brackets, commands and all, it is very easy to make typos.
		\item Plus, \LaTeX~is a really sensitive guy who'll freak out for the smallest things...
		\item Meaning, will not compile the PDF and return a cryptic error message.
		\item Don't panic! (before the first hour is over at least...)
		\item TeXStudio will give you a red sign where it is finding the problem. 
		\item If it was compiling before, redo your last changes. Start debugging from there.
		\begin{itemize}
			\item Reason why it's good practice to compile frequently. Also to save changes.
		\end{itemize}
		\item Or, Google the error message. There are lots of help pages and forums.
	\end{itemize}
\end{frame}

\begin{frame}[fragile]{Getting started}
	\begin{itemize}
		\item Document classes (\verb=\documentclass{}=):
		\begin{itemize}
			\item \cmd{article}: Most common for you. Short document without chapters;
			\item \cmd{report}: longer documents with chapters. Useful for your thesis;
			\item \cmd{book}: for books. Formatting is double-sided with front matter and back matter.
			\item \cmd{letter}: correspondence. Letters of intent.
			\item \cmd{plain}: the most simple, no defaults. Good for debugging.
			\item \cmd{beamer}: presentation slides (as this one...).
		\end{itemize}
		\item Universities and journals often have their own specific templates.
		\item It's the first line in the file.
		\end{itemize}
\end{frame}

\begin{frame}[fragile]{Let's start with an Article}
	\verb=\documentclass{article}=
	\begin{itemize}
		\item We've got to give it a title and an author.
		\item These go into the \textbf{preamble}. That's the chunk of code between \verb=\documentclass{}= and \verb=\begin{document}=. 
		\item Use it to define properties of your file.
		\item Title and author:
		\begin{itemize}
			\item \verb=\title{The Phantom Menace}=
			\item \verb=\author{George Lucas}=
		\end{itemize}
		\item Try printing just a title and your name into a document of class \cmd{article}.
	\end{itemize}
\end{frame}

\begin{frame}[fragile]{If it's not in the Main Body...}
	\begin{verbatim}
	\documentclass{article}
	\title{The Phantom Menace}
	\author{George Lucas}
	\begin{document}
	\maketitle
	\end{document}
	\end{verbatim}
\end{frame}

\begin{frame}[fragile]{Co-authors}
	\begin{verbatim}
	\documentclass{article}
	\title{Big Nature Paper}
	\author{Wrote the Paper \and Supervisor \and Data Owner}
	\begin{document}
		\maketitle
	\end{document}
	\end{verbatim} 
\end{frame}

\begin{frame}[fragile]{Structure}
	\begin{itemize}
		\item A good book or paper should be divided into sections, subsections, etc...
		\item \LaTeX~makes that very easy.
		\item Using
		\begin{itemize}
			\item \verb=\chapter{}=
			\item \verb=\section{}=
			\item \verb=\subsection{}=
			\item \verb=\subsubsection{}=
		\end{itemize}
		\item Will have \LaTeX~take care of the numbering.
	\end{itemize}
\end{frame}

\begin{frame}[fragile]{Try out...}
	\begin{verbatim}
	\documentclass{article}
	\usepackage{lipsum}
	\title{Important Political Science Paper}
	\author{John Nimeta}
	\begin{document}
		\maketitle
		\section{Literature Review}
		\lipsum[1]
		\subsection{Important Findings}
		\lipsum[13]
	\end{document}
	\end{verbatim} 
\end{frame}

\begin{frame}[fragile]{Packages}
	\begin{itemize}
		\item Add-ons, plugins, whatever name you'd be more familiar with...
		\item Thousands of extensions adding functionality.
		\item Expand formatting, style, characters, functions.
		\item Always go in the preamble.
		\item Every \TeX~distribution comes with a set of common packages installed. If you want to use one that is not installed, simply by typing \verb=\usepackage{}=, with the package name within \{\} in the preamble will automatically download and install it.
		\item One of the most common sources of errors... sometimes there are incompatibilities between two packages.
	\end{itemize} 
\end{frame}

\begin{frame}[fragile]{A Few Common Ones}
	\begin{itemize}
		\item \verb=\usepackage[T1]{fontenc}=: Makes some details in the typesetting of fonts look nicer;
		\item \verb=\usepackage{geometry}=: Smaller margins than the default;
		\item \verb=\usepackage{natbib}=: Library of citation styles.
		\item \verb=\usepackage{hyperref}=: Links. Both to outside pages and within the document. 
		\item \verb=\usepackage{setspace}=: Enables alternating between single and double-space between lines.
	\end{itemize}
\end{frame}

\begin{frame}[fragile]{Other Elements in a Paper}
	\begin{itemize}
		\item Abstracts:
	\end{itemize}
	\begin{verbatim}
	\begin{document}
	\maketitle
	\begin{abstract}
	This paper is about...
	\end{abstract}
	\end{document}
	\end{verbatim}
	\begin{itemize}
		\item Footnotes
		\begin{itemize}
			\item \verb=\footnote{}=
		\end{itemize}
	\end{itemize}
\end{frame}

\begin{frame}
	Exercise...
\end{frame}

\begin{frame}[fragile]{Formatting Options}
	\begin{itemize}
		\item First, most \LaTeX~defaults have a reason to be there. Change only if you really want or have to.
		\item A common options...
		\begin{itemize}
			\item \verb=\documentclass[12pt,a4paper]{article}=: font size, family, and paper size are defined as options in the \verb=\documentclass{}= command.
			\item \verb|\usepackage[margin=2cm]{geometry}|: control the size of margins.
			\item \verb|\usepackage{setspace}|, followed by a command \verb|\singlespacing or \doublespacing| in the text to define the space between lines.
		\end{itemize}
		\item Those that go in the preamble affect the whole document.
	\end{itemize}
\end{frame}

\begin{frame}[fragile]{More Formatting Options}
	\begin{itemize}
		\item Options to change formatting only for a word or block of text:
		\begin{itemize}
		\item \verb|\begin{center} Text \end{center}|: centering a block of text.
		\item Font sizes: \tiny{\verb|{\tiny }|}, \scriptsize{\verb|{\scriptsize }|}, \footnotesize{\verb|{\footnotesize }|}, \small{\verb|{\small }|}, \normalsize{\verb|{\normalsize }|}, \large{\verb|{\large }|}, \Large{\verb|{\Large }|}, \LARGE{\verb|{\LARGE }|}, \huge{\verb|{\huge }|}, \Huge{\verb|{\Huge }|}
		\end{itemize}
	\end{itemize}
\end{frame}

\begin{frame}[fragile]{Examples}
	\begin{itemize}
		\item So, the code:
	\end{itemize}
	 \verb|I am a man who \huge walks alone|
	 \begin{itemize}
	 	\item Produces
	 \end{itemize}
	 I am a man who \huge walks alone
	 \begin{itemize}
	 	\item \normalsize And to make
	 \end{itemize}
	 \normalsize I am a man who {\huge walks} alone
	 \begin{itemize}
	 	\item Type
	 \end{itemize}
	 \verb|I am a man who \huge{walks} \normalsize alone|
	 \begin{itemize}
	 	\item or
	 \end{itemize}
	 \verb|I am a man who {\huge walks} alone|
\end{frame}

\begin{frame}[fragile]{Even More Formatting Options}
	\begin{itemize}
		\item We've seen \verb|\textbf{} and \textit{}| already...
		\item further options:
		\begin{itemize}
			\item \verb|\underline|: \underline{Self-explanatory...};
			\item \verb|\textsc|: \textsc{Small caps};
			\item \verb|\emph|: \emph{emphasizing};
		\end{itemize}
	\end{itemize}
\end{frame}

\begin{frame}[fragile]{Vertical Spacing}
	\begin{itemize}
		\item To force small, medium, and large spaces between paragraphs:
		\begin{itemize}
			\item \verb|\smallskip|
			\item \verb|\medskip|
			\item \verb|\bigskip|
		\end{itemize}
		\item Page breaks:
		\begin{itemize}
			\item \verb|\pagebreak|
			\item \verb|\newpage|
		\end{itemize}
		\item Line breaks:
		\begin{itemize}
			\item \textbackslash\textbackslash
		\end{itemize}
	\end{itemize}
\end{frame}

\begin{frame}[fragile]{Environments}
	\begin{itemize}
		\item An environment is a \textit{context};
		\item Different formatting and compiling rules apply within an environment;
		\item We can use them to tell \LaTeX~that what's coming is math, figures, a table, etc.
		\item The commands \verb|\begin{}| and \verb|\end{}| are used to create various environments.
	\end{itemize}
\end{frame}

\begin{frame}[fragile]{Examples of Environments: Enumeration}
	\begin{verbatim}
	Best Christopher Nolan movies
	\begin{enumerate}
	\item The Dark Knight
	\item Inception
	\item Memento
	\end{enumerate}
	\end{verbatim}
	Generates:
	
	Best Christopher Nolan movies
	\begin{enumerate}
		\item The Dark Knight
		\item Inception
		\item Memento
	\end{enumerate}
\end{frame}

\begin{frame}[fragile]{Examples of Environments: Itemize}
	On the other hand,
	\begin{verbatim}
	Best Christopher Nolan movies
	\begin{itemize}
	\item The Dark Knight
	\item Inception
	\item Memento
	\end{itemize}
	\end{verbatim}
	Generates: 
	
		Best Christopher Nolan movies
		\begin{itemize}
			\item The Dark Knight
			\item Inception
			\item Memento
		\end{itemize}
\end{frame}

\begin{frame}[fragile]{Foreign Languages}
	\begin{itemize}
		\item If you have non-English characters in the text file, \LaTeX~will either ignore or freak out...
		\item If we try typing ``\texttt{Universit\"{a}tstra\ss e}'', that's what comes out...
		\begin{itemize}
		\item Universittstrae
		\end{itemize}
		\item If it's just a few special characters in an English text, the cleanest option is using special characters:
		\begin{itemize}
			\item \verb|\"{o}|, \verb|\"{a}|, \verb|\"{u}| producing: \"{o}, \"{a}, \"{u};
			\item \verb|\v{s}|, \verb|\v{c}| or \verb|\v{z}| producing: \v{s}, \v{c}, \v{z};
			\item \verb|\'{a}|, \verb|\'{e}|, \verb|\~{a}|, \verb|\`{a}|, \verb|\^{a}| producing: \'{a}, \'{e}, \~{a}, \`{a}, \^{a}.
		\end{itemize}
		\item And so on... plenty of lists of special characters are available online.
	\end{itemize}
\end{frame}

\begin{frame}[fragile]{Foreign Languages}
	\begin{itemize}
		\item However, for direct input of longer blocks of text, use the \cmd{babel} and \cmd{inputenc} packages:
	\end{itemize}
	\begin{verbatim}
\usepackage[utf8]{inputenc}
\usepackage[russian]{babel}
\begin{document}
Здравствуйте
\end{document}
	\end{verbatim}
	\begin{itemize} 
	\item Generating:
	\end{itemize}
	Здравствуйте
\end{frame}

\begin{frame}[fragile]{Foreign Languages}
	\begin{itemize}
		\item And, for a special section of foreign language text in an English text...
	\end{itemize}
	\begin{verbatim}
	\usepackage[utf8]{inputenc}
	\usepackage[russian,english]{babel}
	\begin{document}
	Populism started with the
	 \foreignlanguage{russian}{народники}, in Russia
	\end{document}
	\end{verbatim}
	\begin{itemize}
		\item Becomes:
	\end{itemize}
	Populism started with the \foreignlanguage{russian}{народники}, in Russia
	\begin{itemize}
		\item NB! The last language in \verb|\usepackage[lang1,lang2]{babel}| is the active.
	\end{itemize}
\end{frame}

\begin{frame}
	Exercise 2
\end{frame}


\end{document}